\newpage
\subsection{Modellierung}\label{sec:modellierung}

In dieser Arbeit geht es darum den Einfluss von Stimmungsdaten bei Aktienprognosen zu untersuchen. Da der Fokus auf den Sentiment-Daten gelegt ist, wird von den Kursdaten lediglich der Schlusspreis einbezogen. Als Basismodell dient also ein LSTM-basiertes Netzwerk, welches Schlusspreise als Eingabe verarbeitet. Das Forschungsmodell nimmt eine Kombination aus Schlusspreis und Stimmungsdaten entgegen.

Insgesamt werden $2$ Experimente durchgeführt. 

Experiment $1$:    
Diese Untersuchung analysiert den Leistungsunterschied zwischen Basis- und Forschungsmodell anhand der GOOG-Aktie.

Experiment $2$:   
Bei dieser Analyse wird der gleiche Unterschied betrachtet, jedoch unter Einbeziehung aller Aktiendaten.

In den nachfolgenden Abschnitten wird zunächst auf modellübergreifende Hyperparameter eingegangen und anschließend erfolgt eine Beschreibung der Architektur von Basis- und Forschungsmodell.

\subsubsection*{Generelle Hyperparameter}\label{sec:modellierung_generell_hp}

Es gibt eine Vielzahl an Hyperparameter, welche sowohl für das Basismodell als auch für das Forschungsmodell gelten. 
In beiden Fällen erfolgt eine Aufteilung der vorverarbeiteten Daten (Kap. \ref{sec:data_prep}) in Trainings-, Validierungs- und Testdaten. Dabei wird $20\%$ der Gesamtmenge für Tests separiert und von den restlichen $80\%$ werden $15\%$ zur Validierung verwendet.
Erst nach diesem Schritt erfolgt die MinMax-Skalierung der Trainingsdaten mit Ausnahme der Embeddings auf das Intervall $0$ bis $1$, mit Hilfe der Python Bibliothek \texttt{sklearn}. Anschließend wird der gelernte Scaler auf Test- und Validierungsdaten angewandt. So wird eine unerwünschten Übertragung von Informationen aus Validierungs- und Testdaten in den Trainingskorpus verhindert. 
Für Kursdaten wird ein Fenster von $30$ vergangenen Markttagen als Modelleingabe verwendet, um den Preis eines nachfolgenden Tages vorherzusagen. Dieser Wert wurde in Experimenten einer Forschungsarbeit an der Universität Shaoguan in China als Optimum ermittelt \footcite[Tabelle 3]{xie2024deep}.
In Anlehnung an eine weitere Forschungsarbeit wird für Stimmungsdaten ein kleineres Fenster von $10$ Markttagen verwendet \footcite[Kap. 4.1]{zhang2022transformer}.
Als Verlustfunktion dient typisch für Regressionsprobleme der \ac{MSE}, welcher sich aus Modellausgabe ($\hat{y}_i$) und tatsächlichem Preis ($y_i$) ergibt (Kap. \ref{sec:theorie_evalmetrics}, Formel \ref{frm:mse}). 
Die Batch-Größe während des Trainings beträgt $64$.



\subsubsection*{Basismodell}\label{sec:modellierung_basis_goog}

D

- lstm \\
- spezifische hyper paramter optimierung mit keras

\subsubsection*{Forschungsmodell}\label{sec:modellierung_forsch}

- Basismodell (etwas abgewandelt) + CNN \\
- spezifische hyper paramter optimierung mit keras \\
- arch unterschied bei experiment 1 und 2 ?








