\newpage
\subsection{Modellierung}\label{sec:modellierung}

In dieser Arbeit geht es darum den Einfluss von Stimmungsdaten bei Aktienprognosen zu untersuchen. Da der Fokus auf den Sentiment-Daten gelegt ist, wird von den Kursdaten lediglich der Schlusspreis einbezogen. Als Basismodell dient also ein LSTM-basiertes Netzwerk, welches Schlusspreise als Eingabe verarbeitet. Das Forschungsmodell nimmt eine Kombination aus Schlusspreis und Stimmungsdaten entgegen.

Insgesamt werden $2$ Experimente durchgeführt. 

Experiment $1$:    
Diese Untersuchung analysiert den Leistungsunterschied zwischen Basis- und Forschungsmodell anhand der GOOG-Aktie.

Experiment $2$:   
Bei dieser Analyse wird der gleiche Unterschied betrachtet, jedoch unter Einbeziehung aller Aktiendaten.

In den nachfolgenden Abschnitten wird zunächst auf modellübergreifende Hyperparameter eingegangen und anschließend erfolgt eine Beschreibung der Architektur von Basis- und Forschungsmodell.

\subsubsection*{Generelle Hyperparameter}\label{sec:modellierung_generell_hp}

- aufteilung training/validierung/testdaten  \\
- skalierung (erst train daten, dann rest)  \\
- zeitfenster (30 tage für kursdaten u. 10 für sent daten)   
- verlustfunktion  


\subsubsection*{Basismodell}\label{sec:modellierung_basis_goog}

- lstm \\
- spezifische hyper paramter optimierung mit keras

\subsubsection*{Forschungsmodell}\label{sec:modellierung_forsch}

- Basismodell (etwas abgewandelt) + CNN \\
- spezifische hyper paramter optimierung mit keras \\
- arch unterschied bei experiment 1 und 2 ?








