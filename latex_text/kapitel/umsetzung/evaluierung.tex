\newpage
\subsection{Evaluierung}\label{sec:evaluierung}

Wie bereits in Kapitel \ref{sec:modellierung} erwähnt, wurden 2 Experimente durchgeführt. Zur Evaluierung werden verschiedene Metriken und Diagramme betrachtet. Dazu gehören zum einen der \ac{MSE} als gewählte Verlustfunktion (Loss) während des Trainings, welcher anhand von skalierten Evaluierungsdaten berechnet wird und zum anderen ein Prognosediagramm und die Metriken \ac{MAE}, \ac{MSE} und \ac{MAPE} (Kap. \ref{sec:theorie_evalmetrics}), welche sich bei der Verwendung von Testdaten ergeben. Für diese Metrik-Berechnungen und das Prognosediagramm werden nicht-skalierte Preise verwendet.
In den nachfolgenden Abschnitten werden die Ergebnisse aufgeführt und analysiert. Im Anschluss daran erfolgt eine Zusammenfassung der gewonnenen Erkenntnisse. 

\subsubsection{Experiment 1: GOOG-Aktie}\label{sec:evaluierung_exp1}
In Experiment $1$ wird das Basismodell mit dem Forschungsmodell verglichen, wobei lediglich Daten der GOOG-Aktie berücksichtigt werden. 

\paragraph*{Verlust-Metrik} 

In Abbildung \ref{fig:vall_loss_exp1} wird die berechnete Verlustentwicklung während des Trainings beider Modelle gegenübergestellt.
\begin{figure}[H]
	\includegraphics[width=1.\textwidth]{vall_loss_exp1}
	\caption{Experiment 1: Verlust-Vergleich von Basis- und Forschungsmodell, berechnet anhand von Validierungsdaten der GOOG-Aktie}
	\label{fig:vall_loss_exp1}
\end{figure}
Die beiden Varianten weisen einen sehr ähnlichen Verlust-Kurvenverlauf auf. Es ist jedoch klar zu erkennen, dass die multimodale Variante leicht bessere Werte erreicht. Ihr Bestwert beträgt $2.14*10^{-2}$, welcher um $4,9\%$ geringer ausfällt als beim Vergleichsmodell.

\paragraph*{Prognose-Metriken} 

In den Abbildungen \ref{fig:exp1_basisprogn} und \ref{fig:exp1_forschprogn} wird der tatsächliche Preis mit der Modellvorhersage verglichen. Die Berechnung der Modellvorhersage erfolgt schrittweise, wobei zur Berechnung des Preises eines Folgetages Realdaten der Vergangenheit an das jeweilige Modell übergeben werden. Demnach erfolgt technisch gesehen keine Langzeitprognose, sondern Tagesprognosen, die auf echten Erfahrungswerten basieren.
\begin{figure}[H]
	\includegraphics[width=1.\textwidth]{exp1_basisprogn}
	\caption{Experiment 1: Basismodell - echter vs. vorhergesagter Preis basierend auf Tagesprognosen}
	\label{fig:exp1_basisprogn}
\end{figure}
\begin{figure}[H]
	\includegraphics[width=1.\textwidth]{exp1_forschprogn}
	\caption{Experiment 1: Forschungsmodell - echter vs. vorhergesagter Preis basierend auf Tagesprognosen}
	\label{fig:exp1_forschprogn}
\end{figure}
Auf Basis der Diagramme ist eine Verbesserung nur schwer erkennbar. Ein Anhaltspunkt ist beispielsweise die Abweichung im Zeitraum ''15-12-22'' bis ''15-12-29''. In Abbildung \ref{fig:exp1_forschprogn} gibt es in diesem Bereich einen deutlich geringeren Anteil an Graufläche, als es bei Abbildung \ref{fig:exp1_basisprogn} der Fall ist.

Anhand der Metrik-Tabelle \ref{tbl:exp1_model_metrics} werden Qualitätsunterschiede deutlicher. Dabei werden die Metriken \ac{MAE}, \ac{MSE} und \ac{MAPE} separat für beide Modelle berechnet.
\begin{table}[H]
	\centering
	\caption{Experiment 1: Metrik-Vergleich zwischen Basis- und Forschungsmodell}
	\label{tbl:exp1_model_metrics}
	\begin{tabular}{lccc}
		\hline
		\textbf{Modell} & \textbf{MAE} & \textbf{MSE} & \textbf{MAPE (\%)} \\
		\hline
		Basis & 14.78 & 357.0 & 2.05 \\
		Forschung & 12.45 & 256.12 & 1.72 \\
		\hline
	\end{tabular}
\end{table}
Die Metriken des Forschungsmodells weisen eine messbare Verbesserung im Vergleich zum Basismodell auf. Dies lässt darauf schließen, dass die Integration von Stimmungsdaten in ein \ac{DL}-Modell für verbesserte Aktienprognosen sorgen kann.

\subsubsection{Experiment 2: Technologie-Aktienmix}\label{sec:evaluierung_exp2}
In Experiment $2$ werden Basis- und Forschungsmodell anhand von Daten mehrerer Aktien der Technologie-Branche trainiert und anschließend anhand von Daten der GOOG-Aktie hinsichtlich Prognose-Qualität getestet.

\paragraph*{Verlust-Metrik} 
In Abbildung \ref{fig:vall_loss_exp2} wird die berechnete Verlustentwicklung während des Trainings beider Modelle gegenübergestellt.
\begin{figure}[H]
	\includegraphics[width=1.\textwidth]{vall_loss_exp2}
	\caption{Experiment 2: Verlustvergleich von Basis- und Forschungsmodell, berechnet anhand der Validierungsdaten von Technologieaktien}
	\label{fig:vall_loss_exp2}
\end{figure}
Bereits das Basismodell weist deutlich bessere Verlustwerte auf als beide Modelle aus Experiment $1$. Das multimodale Forschungsmodell benötigt nun zur Erreichung des Bestwerts mehr Trainingsepochen als das Vergleichsmodell und weist erneut ein leicht besseres Ergebnis auf. Sein minimaler Verlust beträgt $3.2*10^{-3}$ und ist somit um $5,9\%$ kleiner als beim Basismodell.

\paragraph*{Prognose-Metriken} 

Aufgrund der Tatsache, dass die Modelle in diesem Experiment mit Daten mehrerer Aktien trainiert werden, lässt sich für jede involvierte Aktie ein Prognosediagramm erstellen. Für die Vergleichbarkeit mit Experiment $1$ erfolgt eine Fokussierung auf die GOOG-Aktie.
In den Abbildungen \ref{fig:exp2_basisprogn} und \ref{fig:exp2_forschprogn} wird daher erneut, bezogen auf die GOOG-Aktie, der echte Preis mit Tagesprognosen verglichen.
\begin{figure}[H]
	\includegraphics[width=1.\textwidth]{exp2_basisprogn}
	\caption{Experiment 2: Basismodell - echter vs. vorhergesagter Preis basierend auf Tagesprognosen}
	\label{fig:exp2_basisprogn}
\end{figure}
\begin{figure}[H]
	\includegraphics[width=1.\textwidth]{exp2_forschprogn}
	\caption{Experiment 2: Forschungsmodell - echter vs. vorhergesagter Preis basierend auf Tagesprognosen}
	\label{fig:exp2_forschprogn}
\end{figure}
Beim direkten Vergleich beider Diagramme fällt auf, dass in Diagramm \ref{fig:exp2_basisprogn} die Positionen vorhergesagter Preise deutlich stärker in vertikaler Richtung abweichen als in Abbildung \ref{fig:exp2_forschprogn}. Das Forschungsmodell überzeugt also mit Preisvorhersagen nahe am echten Wert.
Mit Hilfe der Metrik-Tabelle \ref{tbl:exp2_model_metrics} werden Qualitätsunterschiede noch deutlicher.
\begin{table}[H]
	\centering
	\caption{Experiment 2: Metrik-Vergleich}
	\label{tbl:exp2_model_metrics}
	\begin{tabular}{lccc}
		\hline
		\textbf{Modell} & \textbf{MAE} & \textbf{MSE} & \textbf{MAPE (\%)} \\
		\hline
		Basis & 10.46 & 187.48 & 1.44 \\
		Forschung & 8.97 & 137.72 & 1.24 \\
		\hline
	\end{tabular}
\end{table}
Wie bereits durch die Verlustmetrik angedeutet, erreicht das Basismodell bessere Werte als beide Modelle aus Experiment $1$. Dies spiegelt sich noch deutlicher in den Testmetriken aus Tabelle \ref{tbl:exp2_model_metrics} wider. Darüber hinaus konnte auch in diesem Fall mit Hilfe des Forschungsmodells das Ergebnis des Basismodells übertroffen werden.

\subsubsection{Erkenntnisse}\label{sec:evaluierung_insights}
Anhand der durchgeführten Experimente lassen sich nützliche Erkenntnisse ableiten, welche in diesem Abschnitt zusammengefasst werden.
Es können zwei Strategien abgeleitet werden, welche eine Verbesserung der Aktienprognose-Qualität ermöglichen. Zum einen die Integration von Stimmungsdaten und zum anderen die Verwendung von Aktienbündeln. Die Verwendung von mehreren Aktien zur Modellierung von Aktienpreisen hat sich als sehr effektiv erwiesen. So konnte die \ac{MSE}-Metrik der Tabelle \ref{tbl:exp1_model_metrics} in Tabelle \ref{tbl:exp2_model_metrics} beim Basismodell um $47,5\%$ und beim Forschungsmodell um $46,2\%$ verbessert werden. Auch die Strategie der Integration von Stimmungsdaten zeigte sich als gewinnbringend. So konnte die \ac{MSE}-Metrik des Basismodells aus Experiment $1$ mit Hilfe des Forschungsmodells um $28,3\%$ verbessert werden und in Experiment $2$ um $26,5\%$.
Den größten Effekt hatte damit die Einbindung von Daten mehrerer Aktien und somit kann eine tiefgründige Extraktion von Informationen bezüglich komplexer Beziehungen zwischen Wertpapieren als sehr gewinnbringend für Aktienprognosen eingestuft werden.
Bei zusätzlicher Hinzunahme von Stimmungsdaten wurden die Ergebnisse noch weiter verbessert (Exp $2$ Forschungsmodell).


