\section{Einleitung}\label{sec:einleitung}

Bereits seid Jahrzenten wird dem Bereich Aktienprognose von Wissenschaftlern als auch Investoren große Aufmerksamkeit gewidmet \footcite[Kap. Introduction]{zhang2022transformer}. Das liegt vor allem daran, dass man mit korrekten Vorhersagen sehr hohe Profite erreichen kann.
Die bisher durchgeführte Forschung hat ergeben, dass numerische Aktien-Daten allein lediglich bis zu einem gewissen Grad zur Verbesserung der Leistung von Deep Learning Modellen beitragen \footcite[Kap. Introduction]{zhang2022transformer}.

Diese Arbeit widmet sich daher der Untersuchung inwieweit Aktienprognosen durch Betrachtung von multimodalen Daten verbessert werden können. Durch den aktuellen technischen Fortschritt gibt es viele Möglichkeiten nicht-numerische Daten einzubinden. Diese Arbeit fokusiert sich auf die Erprobung von vortrainierten großen Sprachmodellen, mit deren Hilfe Stimmungsdaten erzeugt werden sollen. Ein weiterer Schwerpunkt ist die Ermittlung einer geeigneten DL-Architektur, welche mit Numerik- und Textdaten trainiert wird.

Die Strukturierung dieser Arbeit orientiert sich am Crisp-DM Modell. In Kapitel \ref{'sec:kap2'} wird daher zunächst wichtiges Domänenwissen behandelt. Anschließend erfolgt eine Beschreibung der praktischen Umsetzung mitsamt Evaluierung. Darüber hinaus wird auf Möglichkeiten eines Deployment eingegangen. Zum Schluss erfolgt eine zusammenfassende Analyse in Form eines Fazits. 

Das Crisp-DM Modell erfüllt unseren Anspruch an Struktur und Vollständigkeit, wobei vor allem das enthaltene iterative Konzept in unserem Anwendungsfall Vorteile mit sich bringt. Denn falls möglich, soll bei unzureichenden Ergebnissen der Prozessanfang bis Ende auf Verbesserungsmöglichkeiten untersucht werden.  






