\section{Einleitung}\label{sec:einleitung}

Bereits seit Jahrzehnten wird dem Bereich Aktienprognose von Wissenschaftlern wie auch Investoren große Aufmerksamkeit gewidmet \autocite[Kap. Introduction]{zhang2022transformer}. Das liegt vor allem daran, dass man mit korrekten Vorhersagen sehr hohe Profite erreichen kann.

Die bisher durchgeführte Forschung hat ergeben, dass numerische Aktien-Daten allein lediglich bis zu einem gewissen Grad zur Verbesserung der Leistung von \ac{DL}-Modellen beitragen \autocite[Kap. Introduction]{zhang2022transformer}.

Diese Arbeit widmet sich daher der Untersuchung, inwieweit Aktienprognosen durch Betrachtung von multimodalen Daten verbessert werden können. Durch den aktuellen technischen Fortschritt gibt es viele Möglichkeiten, nicht-numerische Daten einzubinden. Diese Arbeit fokussiert sich auf die Erprobung von vortrainierten großen Sprachmodellen, mit deren Hilfe Stimmungsdaten erzeugt werden sollen. Ein weiterer Schwerpunkt ist die Ermittlung einer geeigneten \ac{DL}-Architektur, welche mit Numerik- und Textdaten trainiert wird.

Die Strukturierung dieser Arbeit orientiert sich am \ac{CRISP-DM}-Modell. In Kapitel \ref{sec:theorie} werden zunächst wichtige theoretische Grundlagen behandelt. Anschließend erfolgt eine Beschreibung der praktischen Umsetzung mitsamt Evaluierung. Auf den umfangreichen Arbeitsschritt Deployment wird nur kurz eingegangen, indem wichtige Aspekte im Zusammenhang mit Aktienprognosen beleuchtet werden. Zum Schluss erfolgt eine zusammenfassende Analyse der Ergebnisse in Form eines Fazits und ein Ausblick auf zukünftige Arbeiten. 

Das \ac{CRISP-DM}-Modell erfüllt unseren Anspruch an Struktur und Vollständigkeit, wobei vor allem das enthaltene iterative Konzept in unserem Anwendungsfall Vorteile mit sich bringt. Denn falls möglich, soll bei unzureichenden Ergebnissen der Prozessanfang bis -ende auf Verbesserungsmöglichkeiten untersucht werden.