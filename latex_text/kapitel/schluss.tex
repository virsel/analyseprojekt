\newpage
\section{Fazit \& Zukünftige Arbeiten}
Bei der Trendanalyse ist zu beachten, dass aktuell veröffentlichte Patente (2024-Q2) Innovationen der jüngsten Vergangenheit (bei ungeprüften i.d.R. 18 Monate) darstellen. Die zeitliche Dimension in den Abbildungen \ref{fig:eu_topics_tech}-\ref{fig:ch_topics_tech} und \ref{fig:eu_topics_task}-\ref{fig:ch_topics_task} ist daher im Hinblick auf Innovationsstärke leicht verzerrt.

Die Tabelle \ref{tbl:depatis_treffer} im Kapitel \ref{sec:data_ingestion} macht deutlich, dass China mit Abstand die meisten Innovationen im Robotik-Kontext hervorbringt. Der aktuelle Trend weist jedoch einen plötzlichen, deutlichen Rückgang auf. Diese Entwicklung ist ungewöhnlich und die Gründe dafür können vielfältig sein. Es kann zum Beispiel möglich sein, dass chinesische Unternehmen zunehmend mehr Wert darauf legen, Innovationen im Bereich Robotik und KI vollständig verdeckt zu halten, ähnlich zu der Vorgehensweise von OpenAI mit GPT-4 \autocite{vincent2023openai}.

Europa ist, wie zu erwarten, das Schlusslicht. Auch nach Anwendung eines, auf Beschäftigtenzahlen je Region basierenden, Äquivalenzfaktors blieb die Überlegenheit Chinas in nahezu allen Bereichen bestehen (Abb. \ref{fig:ipc_topic_distr}, \ref{fig:tech_topicdistr} u. \ref{fig:task_topicdistr}). 
Die Analysen ergaben, dass die USA im Bereich Medizin und Teleoperation im Vergleich zu Europa und China am intensivsten an Neuerungen gearbeitet hatten, so dass dort der größte gewichtete Anteil an Patenten veröffentlicht wurde.

Die Überlegenheit Chinas bei der Patentanzahl muss nicht zwingend bedeuten, dass dieses Land die größte Innovationskraft aufweist. In dieser Arbeit wurden veröffentlichte Patente einbezogen, aber deren Neuartigkeit ist nicht zwangsläufig geprüft und auch eine Bewertung des Einfallsreichtums ist nicht gegeben. So kann es durchaus sein, dass in den USA bei Patenten mehr Wert auf Qualität bzw. Einfallsreichtum der Erfindung gelegt wird, so dass banalere Neuerungen bei der Prüfung abgelehnt werden.