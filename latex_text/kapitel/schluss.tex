\newpage
\section{Auswertung}
Bei der Trendanalyse ist zu beachten, dass aktuell veröffentlichte Patente (2024-Q2) Innovationen der jüngsten Vergangenheit (bei ungeprüften i.d.R. 18 Monate) darstellen. Die zeitliche Dimension in den Abbildungen \ref{fig:eu_topics_tech}-\ref{fig:ch_topics_tech} und \ref{fig:eu_topics_task}-\ref{fig:ch_topics_task} ist daher im Hinblick auf Innovationsstärke leicht verzerrt.

Die Tabelle \ref{tbl:depatis_treffer} im Kapitel \ref{sec:data_ingestion} macht deutlich, dass China mit Abstand die meisten Innovationen im Robotik-Kontext hervorbringt. Der aktuelle Trend weist jedoch einen plötzlichen, deutlichen Rückgang auf. Diese Entwicklung ist ungewöhnlich und die Gründe dafür können vielfältig sein. Es kann zum Beispiel möglich sein, dass chinesische Unternehmen zunehmend mehr Wert darauf legen, Innovationen im Bereich Robotik und KI vollständig verdeckt zu halten, ähnlich zu der Vorgehensweise von OpenAI mit GPT-4 \footcite{vincent2023openai}.

Europa ist, wie zu erwarten, das Schlusslicht. Auch nach Anwendung eines, auf Beschäftigtenzahlen je Region basierenden, Äquivalenzfaktors blieb die Überlegenheit Chinas in nahezu allen Bereichen bestehen (Abb. \ref{fig:ipc_topic_distr}, \ref{fig:tech_topicdistr} u. \ref{fig:task_topicdistr}). 
Die Analysen ergaben, dass die USA im Bereich Medizin und Teleoperation im Vergleich zu Europa und China am intensivsten an Neuerungen gearbeitet hatten, so dass dort der größte gewichtete Anteil an Patenten veröffentlicht wurde.

Die Überlegenheit Chinas bei der Patentanzahl muss nicht zwingend bedeuten, dass dieses Land die größte Innovationskraft aufweist. In dieser Arbeit wurden veröffentlichte Patente einbezogen, aber deren Neuartigkeit ist nicht zwangsläufig geprüft und auch eine Bewertung des Einfallsreichtums ist nicht gegeben. So kann es durchaus sein, dass in den USA bei Patenten mehr Wert auf Qualität bzw. Einfallsreichtum der Erfindung gelegt wird, so dass banalere Neuerungen bei der Prüfung abgelehnt werden.


\subsection*{Zukünftige Arbeiten}\label{sec:future_work}
Die Auswertung führte zur Offenlegung von Limitationen, welche in zukünftigen Arbeiten ausgebessert werden. 
Ein wichtiger Schritt zur genaueren Vergleichbarkeit der Innovationskraft ist die Eingrenzung der Patente auf bereits Geprüfte. Außerdem soll eine Methode zur Bewertung des Einfallsreichtums der Erfindung umgesetzt werden, denn eine sehr gute Idee kann für mehr Innovation sorgen als viele moderate. In dieser Arbeit wurden lediglich Patentveröffentlichungen aus dem \ac{DEPATIS} Dokumentenarchiv einbezogen. Ein weiterer wichtiger Schritt ist eine Ausweitung der Datenquellen. Dabei sollen mindestens all jene Quellen einbezogen werden, welche Bezug zu den involvierten Regionen aufweisen. Zudem sollen auch wissenschaftliche Arbeiten als Indikator für Innovationskraft hinzugefügt werden. Auch eine Erweiterung beziehungsweise Anpassung der einbezogenen Regionen wird für mehr und verlässlicheren Informationsgehalt der Analysen sorgen.

\newpage
\section{Fazit}
Im Hinblick auf die Forschungsfrage lässt sich zusammenfassen, dass die untersuchten Regionen in Bezug auf Schwerpunktsetzung viele Gemeinsamkeiten aufweisen und sich nur in Nuancen unterscheiden. Erschreckend, aber nicht überraschend war der interkontinentale Fokus von Robotik mit Bezug zu militärischen Zwecken, wobei dieser in China am ausgeprägtesten ist. 
Als aktuell innovativste Region, welche zudem auch am intensivsten Patente veröffentlicht, wurde mit deutlichem Vorsprung China ermittelt. Es ist also sehr wahrscheinlich, dass in Zukunft bahnbrechende, massentaugliche Produkte mit Robotik-Bezug auf dem Markt erscheinen, welche zum Großteil auf Innovationen aus China basieren. Die USA haben in den Bereichen Teleoperation und Gesundheit eine dominierende Innovationsintensität bewiesen. Der erhöhte Fokus dieser Region auf Robotik im medizinischen Bereich sollte von anderen Regionen zum Vorbild genommen werden, wobei Europa eine positive Tendenz dahingehend aufweist.
