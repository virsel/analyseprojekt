\newpage
\section{Fazit \& Ausblick}
Abschließend erfolgt in diesem Kapitel eine übergeordnete Analyse der Ergebnisse hinsichtlich Anwendbarkeit der entwickelten Lösung. Es erfolgt eine genaue Betrachtung der bestehenden Möglichkeiten und Limitationen. Abgeleitet davon erfolgt im letzten Abschnitt ein Ausblick auf mögliche Erweiterungen durch zukünftige Arbeiten. 

\subsection{Möglichkeiten}\label{sec:evaluierung_chance}
Die Ergebnisse aus Kapitel \ref{sec:evaluierung} zeigen deutlich, dass \ac{DL}-Modelle in der Lage sind, komplexe Muster in Kursdaten und Stimmungsdaten zu erlernen, welche präzise Aktienprognosen unterstützen. Das Forschungsmodell aus Experiment $2$ ist in der Lage, eine Tagesprognose mit einem \ac{MAPE} von $1.24\%$ zu liefern. Je nachdem, ob diese Vorhersage des Folgepreises einen Anstieg oder Abfall darstellt, lässt sich davon eine Handelsentscheidung ableiten. Diese sollte jedoch nicht ausschließlich auf Basis der Prognose getroffen werden, sondern in Kombination mit weiteren Informationen und Analysen. Ähnlich wie die Technik Ensemble-Learning aus dem Machine-Learning-Bereich könnte die Verwendung von mehreren Modellen das Risiko einer Fehlentscheidung reduzieren \autocite{wu2021ensemble}. Außerdem ist es wichtig, dass man sich der Limitierungen des Modells bewusst ist und diese bei der Handelsentscheidung berücksichtigt.


\subsection{Limitationen}\label{sec:evaluierung_limits}
Das Ziel von Aktienprognosen mit Hilfe von Deep Learning besteht natürlich darin, anhand von Modellausgaben auf lange Sicht am Aktienmarkt Gewinne zu realisieren. Die Modelle aus Experiment $1$ sind aufgrund der hohen Anforderung an den Umfang der Trainingsdaten lediglich auf bereits langjährig etablierte Aktien anwendbar. Experiment $2$ lässt sich in einer nachfolgenden Arbeit dahingehend erweitern, dass Daten etablierter und neuer Aktien gemischt für das Training verwendet werden. Doch in beiden Fällen ist für eine optimale Vorhersage das Bestehen der Aktie seit 30 Markttagen in die Vergangenheit notwendig. 
Eine weitreichendere Limitation besteht darin, dass die Preisvorhersagen eine Trägheit aufweisen. Das heißt, dass in Phasen von Preisanstieg höchstwahrscheinlich auch ein weiterer Anstieg prognostiziert wird. Die Modelle weisen deutlich Schwierigkeiten bei der Vorhersage von Richtungsänderungen auf. Doch genau diese Fähigkeit ist für ein erfolgreiches Handeln am Aktienmarkt wichtig. Ein Grund dafür könnte sein, dass im Bereich der Sentiment-Daten alle Tweets mit gleichem Gewicht in die Modelle einfließen. Dies spiegelt nicht die Realität wider, da Tweets von hochrangigen Experten oft mehr Einfluss auf die Marktentwicklung nachweisen als Tweets von Laien.
Eine weitere substanzielle Limitation besteht darin, dass lediglich der Preis eines Folgetages anhand von echten Preisen 30 vorangegangener Markttage vorhergesagt wurde. Die Modelle sind nicht auf mehrtägige Prognosen optimiert und wurden nicht dahingehend getestet.  

\subsection{Ausblick}\label{sec:evaluierung_ausblick}
Um eine Verbesserung der Vorhersagequalität durch Integration von Stimmungsdaten zu erreichen, sollte im Beispiel von Tweets die Tragweite der Aussagen abhängig vom Tweet-Ersteller und seiner Reichweite gewichtet werden. Darüber hinaus sollten weitere Informationsquellen wie News, Unternehmensberichte oder Interviews in die Analyse einbezogen werden. Ein aktuelles Beispiel für Kursänderung aufgrund von Expertenmeinungen ist eine Q\&A-Runde mit dem Nvidia-CEO Jensen Huang und Analysten, in der sich Jensen pessimistisch gegenüber der Zukunft von Quantencomputern geäußert hatte. Dies führte zu einem sprunghaften Kursrückgang von zahlreichen Quantenaktien, welcher sich anschließend allmählich wieder erholte \autocite{ntv2025nvidia}\autocite{finanzen2025microsoft}.
Des Weiteren sollte die Datenqualität optimiert werden, indem beispielsweise inhaltslose Tweets herausgefiltert werden. Außerdem kann die Verwendung eines großen Sprachmodells wie GPT-4o oder Deepseek-V3 zur Extraktion von Stimmungsdaten die Vorhersage-Qualität verbessern \autocite{urldeepseek}\autocite{urlchatgpt}.
Ein weiterer wichtiger Aspekt, den es zu vertiefen gilt, ist die Analyse von Beziehungen zwischen verschiedenen Aktien, um so eine bestmögliche Zusammenstellung mehrerer Aktien für das Training zu erreichen. Dies könnte durch die Verwendung von Graph-Neural-Networks realisiert werden, welche in der Lage sind, Beziehungen zwischen Knoten zu erlernen \autocite{qian2024mdgnn}\autocite[Kap. 6]{zhang2022transformer}. Diese Strategie ist auch ein wichtiger Schritt, um zuverlässige Prognosen für neuere Aktien zu ermöglichen.




