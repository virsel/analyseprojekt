\section{Einleitung}\label{sec:einleitung}

Die Entwicklung der Rechenleistung von Maschinen hat in den letzten Jahrzehnten bemerkenswerte Fortschritte gemacht und ermöglichte damit den unwiderruflichen Durchbruch von KI-Systemen. Eine Studie der Harvard Business School ergab, dass eine Verwendung von Sprachmodellen als Hilfsmittel bei Wissensarbeiten zur deutlichen Steigerung der Produktivität und Ergebnisqualität führt \footcite{prod_llm_study}. Bezüglich des Zukunftspotenzials von generativer KI gibt es eine Studie von McKinsey \& Company, in welcher prognostiziert wird, dass im Bereich Forschung und Entwicklung mit einer jährlichen Produktivitätssteigerung im Wert von $10-15\%$ der bisherigen Kosten zu rechnen ist \footcite{study_mckinsey}. Vor allem auch der Bereich Robotik wird mit Sicherheit in Zukunft zunehmend die Arbeitslast von Menschen verringern. Roboter gibt es bereits in vielen verschiedenen Bereichen, doch zur Zeit ist deren Einsatz noch beschränkt auf spezielle Anwendungsfälle wie Fertigungsstraßen, Logistikzentren und einfache Haushaltsaufgaben. Die bereits zum Einsatz kommenden Technologien, die einst als futuristisch galten, sind in vielen Branchen unserer Gesellschaft unentbehrlich geworden. Robotik-Entwickler profitieren von verfügbaren KI-Technologien, so dass große Durchbrüche von Robotern mit erweiterten Fähigkeiten in naher Zukunft realistisch sind. 

Diese Arbeit beschäftigt sich mit der Forschungsfrage, inwieweit sich weltweit einflussreiche Industrieregionen bei der Robotik-Entwicklung im Hinblick auf Innovationskraft, Intensität und Schwerpunktsetzung unterscheiden. 
Dies soll dazu beitragen, ein besseres Verständnis für die globalen Dynamiken und die geografischen Unterschiede bei der Innovation im Bereich der Robotik zu schaffen. Die aus dieser Arbeit hervorgehenden Erkenntnisse sollen nicht nur zur wissenschaftlichen Diskussion beitragen, sondern auch potenziellen Investoren und politischen Entscheidungsträgern wichtige Hinweise auf die strategische Ausrichtung und Prioritäten der technologischen Forschung und Entwicklung liefern. Die Analysen können zum Beispiel Unternehmen bei fundierten Entscheidungen bezüglich Kooperationspartnern und Standorten unterstützen.

Vor allem die Länder China und USA sind um Führungspositionen auf dem Robotik-Markt bemüht, wobei aktuell humanoide Roboter im Fokus liegen \footcite{chinavsusa}. Auch Europa hat gute Chancen in solchen Zukunftsmärkten vorne mitzuspielen \footcite{gervsusa}. Dieses Vorwissen lässt bereits eine Hypothese zum möglichen Ausgang dieser Arbeit aufstellen. Durch die jüngsten Fortschritte von Unternehmen aus den USA, wie zum Beispiel OpenAI, und das vorherrschende Potenzial in diesem Land wird davon ausgegangen, dass China und die USA hinsichtlich Innovationskraft nah beieinander liegen. Vermutlich weist China aufgrund seiner Überlegenheit hinsichtlich der Bevölkerungszahl leichte Vorteile auf. Bezüglich Diversität im Bereich Technologie und Einsatzzweck von Robotik fällt es schwer eine Prognose für die Schwerpunktsetzung festzulegen. Diese wird wahrscheinlich regional ähnlich ausfallen, wobei mit einer hohen Innovationskraft bezüglich der Einsatzzwecke Logistik und Handwerk zu rechnen ist.

Um den Rahmen der zu untersuchenden Regionen überschaubar abzugrenzen, beschränkt sich die Untersuchung auf China und USA als führende Repräsentanten der Kontinente Ostasien und Nordamerika. Zusätzlich werden dominierende Vertreter aus Europa kumuliert.

Als Datengrundlage dienen Patentdaten, wobei angenommen wird, dass man von diesen technologische Fortschritte und eine strategische Fokussierung ableiten kann. Die Annahme basiert darauf, dass in der internationalen Forschung und Entwicklung der Schutz geistigen Eigentums durch Patente eine entscheidende Rolle spielt. Durch eine Analyse von Patentdaten soll daher ein tiefer Einblick in die Innovationskraft und spezifischen Trends ermöglicht werden.

Für die Umsetzung der Analysen werden quantitative und qualitative Aspekte der Patentdaten berücksichtigt, wobei als Methodik eine Vorgehensweise gemäß dem \ac{CRISP-DM}-Modell gewählt wird. Der Vorteil dieser Methode liegt in ihrer Struktur und Flexibilität, systematisch und iterativ vorzugehen \footcite{website:ibm_crispdm}.




