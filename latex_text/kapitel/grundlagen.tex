\newpage
\section{Theoretische Grundlagen}\label{sec:theorie}



\subsection{Datenverständnis}\label{sec:theorie_data_understanding}

---

Here's the German translation:
Im Finanzbereich werden Aktien in 9 Branchen kategorisiert: Grundstoffe, Konsumgüter, Gesundheitswesen, Dienstleistungen, Versorgungsunternehmen, Mischkonzerne, Finanzwesen, Industriegüter und Technologie. Da Aktien mit hohem Handelsvolumen tendenziell häufiger auf Twitter diskutiert werden, wählen wir die zweijährigen Kursbewegungen von 88 Aktien vom 01.01.2014 bis 01.01.2016 als Ziele aus, bestehend aus allen 8 Aktien der Mischkonzerne und den Top 10 Aktien nach Kapitalvolumen aus jeder der anderen 8 Branchen (siehe ergänzendes Material).

---
\footcite[Kap. 3]{xu2018StockMovement}

\subsection{Datenvorverarbeitung mit Pandas}\label{sec:theorie_pandas}


\subsection{Aktienprognosen mit Deep Learning}\label{sec:theorie_dl}

\subsubsection{Modellierung mit Keras}\label{sec:theorie_keras}

\subsubsection{LSTM für Kursdaten}\label{sec:theorie_lstm}

\subsubsection{LLM für Stimmungsdaten}\label{sec:theorie_llm}
---

FinancialBERT applies domain-specific language understanding to financial text analysis. Built by ahmedrachid, this model stands alongside other financial sentiment analyzers like finbert-tone and finbert. The model was fine-tuned on the Financial PhraseBank dataset, achieving 98\% weighted average precision across sentiment categories.
...

---
\footcite{hazourli2022financialbert}


\subsubsection{CNN für Stimmungsdaten}\label{sec:theorie_cnn}

\subsubsection{Evaluierungsmetriken}\label{sec:theorie_evalmetrics}

- \ac{MAE}, \ac{MSE} und \ac{MAPE}  
\footcite[Kap. 4.3]{xie2024deep}

mse:

\begin{formel}[h]
	\caption{\ac{MSE}}
	\label{frm:mse}
	\begin{align}
		MSE = \frac{1}{N} \sum_{i=1}^{N} (y_i - \hat{y}_i)^2
	\end{align}
	\vspace{0.5em}
	\normalsize{Quelle: ...}
	\vspace{-1.0em}
\end{formel}







